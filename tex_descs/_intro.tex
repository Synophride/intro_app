\section{Introduction}
Ce projet porte sur
l'étiquetage morpho-syntaxique, dit Part of Speach, d'un
corpus.

\subsection{Part of speech}
L'étiquetage morpho-syntaxique est l'association à chaque mot d'un
texte des informations grammaticales liées à ce même mot.


Dans le cadre de ce projet, les informations grammaticales se limitent
aux informations liés à la nature grammaticale des mots. Les natures
possibles sont les suivantes: 


\begin{description} 
\item[ADJ] Les adjectifs qui caractérisent le nom et s'accordent en genres et en nombres. 
\item[ADV] Les adverbes qui précisent le sens d'un verbe et marquent le degré d'intensité de la phrase. 
\item[INTJ] Les interjections qui montrent l'exclamation. 
\item[NOUN] Les noms qui désignent une réalité concrète ou abstraite et s'accordent en genre et en nombre. 
\item[PROPN] Les noms propres qui désignent le nom d'un individu, un lieu ou un objet spécifique. 
\item[VERB] Les verbes sont les noyaux de la phrase, ils expriment un état ou une action et varient en personne. 
\item[DET] Les déterminants précèdant un nom et qui marquent le genre et le nombre de celui-ci. 
\item[NUM] Des nombres
\item[PRON] Les pronoms qui désignent une personne et remplacent un élément déjà nommé. 
\item[CCONJ] Les conjonctions de coordination qui permettent de relier deux mots, propositions et phrases. 
\item[AUX] Les auxiliaires qui sont des verbes qui se combinent à un verbe principal afin de constituer un temps composé, exprimant le temps, le mode et la voix. 
\item[ADP] Les prépositions qui permettent de relier un nom, verbe ou adjectif à un nom, pronom, adverbe ou à un verbe. 
\item[SCONJ] Les conjonctions de subordination permettant d'introduire une proposition subordonnée. 
\item[PART] Les particules qui permettent de donner un sens une fois associés à un mot. 
\item[PUNCT] Les ponctuations qui permettent l'organisation de l'écrit en indiquant les pauses et intonations ou en précisant le sens de la phrase. 
\item[SYM] Les symboles qui sont des mots spéciaux différant des mots
  ordinaires par leurs forme, fonction ou les deux.  
\item[X] Les autres qui sont des mots ne pouvant être assignés aux autres classes grammaticales. 
\end{description}


\subsection{Objectif}
Notre objectif est d'implémenter et d'entraîner
un neurone artificiel afin de prédire
la nature de chaque mot, connaissant son contexte
(égal à la phrase dans lequel il se trouve).


Pour cela, nous disposons de plusieurs jeux de données provenant pour
la plupart de la base \emph{ Universal Dependencies }. Nous avons
écarté les \emph{dataset} en anglais pour nous concentrer sur les
données écrites en Français. Ces données seront décrites plus en
détail dans la deuxième partie.


\subsection{Plan}
Dans un premier temps, nous allons décrire les données, et montrer
quelques informations sur ces dernières. \\
Ensuite, nous présenterons la structure et le contenu du code, ainsi
que les perceptrons et la modélisation choisie \\
Nous montrerons et expliquerons les résultats obtenus,
avant de conclure.


%Grâce à ces données, nous avons voulu créer un perceptron utilisant
%différents modèles pouvant prédire l'étiquetage morpho-syntaxique des
%mots dans les  phrases. Nous avons aussi voulu construire plusieurs
%fonctions d'analyses afin d'étudier les données à notre dispositions. 
