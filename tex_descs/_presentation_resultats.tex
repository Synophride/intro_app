\section{Résultats}
Dans un premier temps, voici les résultats des modèles sur le
dataset {\verb ftb }

\begin{figure}[H] ·
  \begin{adjustbox}{width=\textwidth}
    
  \begin{tabular}{ |l|| *{5}{c|} }\hline
    
Classifieur   & Modele tp5 & Perceptron nodistrib & Perceptron distrib
& SKLearn Perceptron & SKLearn SVM \\ \hline
Pourcentage de réussite   & 94.70 & 93.97 & 93.47 & 94.13 & 96.71 \\ \hline
  \end{tabular}
    \end{adjustbox}
  \caption{Taux de réussite sur les données pour divers classifieurs}

\end{figure}




Nous remarquons que :
\begin{itemize}
\item Les résultats sont bons sur cet ensemble de données (avec
  plus de 90 \% à chaque fois.
  Cela reste à relativiser, étant donné que ces données d'entraînement sont
  de loin les plus conséquentes, avec plus de 14000 phrases.
\item La construction du dictionnaire nécessaire aux
  \emph{distributional features} prend beaucoup de temps.
\item Les \emph{distributional features} que l'on a construit
  ne servent à rien; au contraire elles baissent le score de réussite sur
  le \emph{dataset} {\verb ftb }.
  Cela est probablement dû à une mauvais compréhension de l'article,
  ayant mené à une mauvaise implémentation.
\end{itemize}

Maintenant, nous allons nous intéresser à des tableaux de résultats
plus détaillés. Pour chacun de ces tableaux, chaque colonne représente
un ensemble d'entraînement, tandis que chaque ligne représente un
ensemble de test. 


Pour chaque ensemble d'entraînement de de test, nous avons trois
mesures, toutes exprimées en pourcentage :
\begin{description}
\item[OOV] La réussite sur les mots hors vocabulaire, ie les mots
  apparaissant dans l'ensemble de test mais pas dans l'ensemble
  d'entrainement.
\item[AMB] La réussite sur les mots ambigus, qui apparaissent dans les
  données de train ou de test, et qui peuvent avoir deux natures
  différentes
\item[GEN] Le taux de réussite général.
\end{description}

\begin{figure}[H] \begin{adjustbox}{width=\textwidth} \begin{centering} \begin{tabular}{ | l || *{ 6}{c|c|c||} } \hline 
& \multicolumn{3}{|c|}{ ftb } & \multicolumn{3}{|c|}{ gsd } & \multicolumn{3}{|c|}{ partut } & \multicolumn{3}{|c|}{ pud } & \multicolumn{3}{|c|}{ sequoia } & \multicolumn{3}{|c|}{ spoken }  \\ \hline 
& OOV & AMB & GEN & OOV & AMB & GEN & OOV & AMB & GEN & OOV & AMB & GEN & OOV & AMB & GEN & OOV & AMB & GEN   \\ \hline \hline 
foot  & 25.46 & 82.28 & 65.95
 & 22.34 & 77.01 & 67.94
 & 42.11 & 73.30 & 61.04
 & 42.11 & 73.30 & 61.04
 & 33.37 & 78.45 & 64.89
 & 31.63 & 69.44 & 50.65
 \\ \hline 
ftb  & 76.31 & 93.01 & 94.70
 & 70.97 & 90.93 & 91.01
 & 69.94 & 86.36 & 85.29
 & 69.94 & 86.36 & 85.29
 & 76.85 & 91.23 & 90.36
 & 44.26 & 71.82 & 62.65
 \\ \hline 
gsd  & 67.61 & 89.03 & 89.99
 & 78.81 & 93.51 & 94.62
 & 69.04 & 88.87 & 85.36
 & 69.04 & 88.87 & 85.36
 & 76.96 & 91.59 & 90.02
 & 45.01 & 75.84 & 64.46
 \\ \hline 
natdis  & 24.92 & 87.98 & 78.52
 & 46.93 & 86.72 & 79.10
 & 51.74 & 78.38 & 71.35
 & 51.74 & 78.38 & 71.35
 & 51.72 & 84.71 & 76.22
 & 31.10 & 69.26 & 55.00
 \\ \hline 
partut  & 74.03 & 87.81 & 90.81
 & 71.73 & 92.37 & 91.84
 & 69.28 & 90.81 & 91.13
 & 69.28 & 90.81 & 91.13
 & 79.17 & 92.70 & 90.13
 & 50.87 & 75.71 & 69.22
 \\ \hline 
pud  & 61.45 & 81.50 & 84.84
 & 67.53 & 84.02 & 87.96
 & 67.81 & 79.07 & 80.64
 & 67.81 & 79.07 & 80.64
 & 72.38 & 80.88 & 83.76
 & 47.75 & 75.58 & 64.63
 \\ \hline 
sequoia  & 70.82 & 90.18 & 91.06
 & 73.17 & 93.20 & 92.73
 & 70.61 & 90.15 & 85.69
 & 70.61 & 90.15 & 85.69
 & 81.31 & 94.37 & 94.64
 & 46.44 & 75.60 & 65.06
 \\ \hline 
spoken  & 49.55 & 71.92 & 78.64
 & 51.09 & 77.00 & 82.50
 & 60.32 & 75.70 & 76.85
 & 60.32 & 75.70 & 76.85
 & 63.71 & 73.80 & 78.02
 & 78.57 & 86.65 & 90.34
 \\ \hline 
 \end{tabular} \end{centering} \end{adjustbox} \caption{ Taux de réussite détaillés pour le perceptron Modele tp5} \end{figure} 

De manière générale, le score de 90\% sur la réussite des données
était flatteur, la plupart des résultats tournant autour de 80\%

Nous remarquons que les scores les plus bas sur les tests
proviennent des ensembles de test natdis et foot
(resp. autour de 80 et 70 \% sur la plupart des corpus).


Deux explications pourraient exister : La première étant que, de
manière générale, les tweets sont rarement écrits dans un français
correct,
la seconde serait que ces derniers contiennent
un grand nombre de labels spécifiques
(par exemple les mentions) à twitter.


Par conséquent, des scores particulièrement bas sur ces \emph{dataset}
se retrouveront avec les autres classifieurs.


\begin{figure}[H] \begin{adjustbox}{width=\textwidth} \begin{centering} \begin{tabular}{ | l || *{ 6}{c|c|c||} } \hline 
& \multicolumn{3}{|c|}{ ftb } & \multicolumn{3}{|c|}{ gsd } & \multicolumn{3}{|c|}{ partut } & \multicolumn{3}{|c|}{ pud } & \multicolumn{3}{|c|}{ sequoia } & \multicolumn{3}{|c|}{ spoken }  \\ \hline 
& OOV & AMB & GEN & OOV & AMB & GEN & OOV & AMB & GEN & OOV & AMB & GEN & OOV & AMB & GEN & OOV & AMB & GEN   \\ \hline \hline 
foot  & 12.98 & 79.61 & 59.18
 & 18.69 & 76.62 & 63.53
 & 18.90 & 76.33 & 46.63
 & 18.90 & 76.33 & 46.63
 & 21.33 & 78.00 & 57.85
 & 20.39 & 72.79 & 43.13
 \\ \hline 
ftb  & 61.13 & 92.70 & 93.47
 & 62.01 & 90.52 & 89.05
 & 51.37 & 84.32 & 75.70
 & 51.37 & 84.32 & 75.70
 & 58.72 & 90.26 & 83.50
 & 33.64 & 73.98 & 55.87
 \\ \hline 
gsd  & 47.98 & 89.35 & 87.30
 & 65.97 & 92.43 & 91.93
 & 46.12 & 85.90 & 75.09
 & 46.12 & 85.90 & 75.09
 & 56.81 & 89.96 & 82.50
 & 33.08 & 76.76 & 57.20
 \\ \hline 
natdis  & 16.68 & 86.56 & 75.15
 & 36.73 & 85.83 & 75.97
 & 28.52 & 78.70 & 57.04
 & 28.52 & 78.70 & 57.04
 & 34.05 & 86.82 & 66.93
 & 23.25 & 74.80 & 50.05
 \\ \hline 
partut  & 56.73 & 89.89 & 89.78
 & 58.69 & 89.98 & 88.07
 & 49.82 & 86.08 & 81.55
 & 49.82 & 86.08 & 81.55
 & 57.72 & 90.70 & 80.79
 & 38.49 & 75.71 & 61.15
 \\ \hline 
pud  & 41.10 & 80.53 & 81.12
 & 55.42 & 83.10 & 85.25
 & 45.70 & 79.59 & 70.91
 & 45.70 & 79.59 & 70.91
 & 51.88 & 80.47 & 76.15
 & 34.36 & 78.40 & 57.09
 \\ \hline 
sequoia  & 54.20 & 91.51 & 88.81
 & 59.69 & 92.32 & 89.77
 & 45.03 & 86.12 & 74.57
 & 45.03 & 86.12 & 74.57
 & 60.51 & 92.48 & 88.03
 & 34.75 & 76.94 & 57.51
 \\ \hline 
spoken  & 36.09 & 71.55 & 76.28
 & 33.71 & 76.72 & 79.42
 & 35.75 & 73.76 & 65.82
 & 35.75 & 73.76 & 65.82
 & 44.16 & 72.65 & 70.95
 & 57.03 & 85.13 & 83.21
 \\ \hline 
 \end{tabular} \end{centering} \end{adjustbox} \caption{ Taux de réussite détaillés pour le perceptron Perceptron distrib} \end{figure} 
\begin{figure}[H] \begin{adjustbox}{width=\textwidth} \begin{centering} \begin{tabular}{ | l || *{ 6}{c|c|c||} } \hline 
& \multicolumn{3}{|c|}{ ftb } & \multicolumn{3}{|c|}{ gsd } & \multicolumn{3}{|c|}{ partut } & \multicolumn{3}{|c|}{ pud } & \multicolumn{3}{|c|}{ sequoia } & \multicolumn{3}{|c|}{ spoken }  \\ \hline 
& OOV & AMB & GEN & OOV & AMB & GEN & OOV & AMB & GEN & OOV & AMB & GEN & OOV & AMB & GEN & OOV & AMB & GEN   \\ \hline \hline 
foot  & 15.14 & 81.72 & 60.80
 & 18.90 & 78.05 & 65.22
 & 20.48 & 66.79 & 47.04
 & 20.48 & 66.79 & 47.04
 & 17.24 & 70.18 & 52.36
 & 16.49 & 71.76 & 41.19
 \\ \hline 
ftb  & 61.40 & 93.28 & 93.97
 & 62.36 & 90.45 & 89.44
 & 56.19 & 83.41 & 79.83
 & 56.19 & 83.41 & 79.83
 & 59.87 & 89.68 & 84.71
 & 37.53 & 73.12 & 58.34
 \\ \hline 
gsd  & 52.01 & 89.44 & 88.20
 & 67.18 & 92.20 & 92.34
 & 52.81 & 84.55 & 79.00
 & 52.81 & 84.55 & 79.00
 & 54.46 & 90.32 & 82.81
 & 34.41 & 77.48 & 58.83
 \\ \hline 
natdis  & 15.48 & 85.76 & 75.01
 & 33.95 & 86.35 & 75.96
 & 35.83 & 77.01 & 61.19
 & 35.83 & 77.01 & 61.19
 & 42.64 & 83.86 & 68.51
 & 23.64 & 74.96 & 51.35
 \\ \hline 
partut  & 50.96 & 89.73 & 89.86
 & 55.43 & 90.60 & 88.50
 & 58.02 & 87.56 & 87.51
 & 58.02 & 87.56 & 87.51
 & 60.88 & 90.58 & 84.73
 & 46.42 & 75.90 & 65.16
 \\ \hline 
pud  & 44.77 & 80.88 & 82.04
 & 55.36 & 83.80 & 85.91
 & 50.21 & 73.78 & 73.49
 & 50.21 & 73.78 & 73.49
 & 48.99 & 79.72 & 76.50
 & 36.96 & 77.25 & 59.09
 \\ \hline 
sequoia  & 53.51 & 91.04 & 88.61
 & 59.81 & 91.63 & 89.90
 & 50.71 & 87.13 & 78.54
 & 50.71 & 87.13 & 78.54
 & 61.84 & 92.70 & 89.49
 & 38.14 & 75.87 & 59.79
 \\ \hline 
spoken  & 35.87 & 70.77 & 76.30
 & 33.33 & 75.47 & 78.53
 & 36.93 & 69.03 & 65.81
 & 36.93 & 69.03 & 65.81
 & 47.67 & 71.04 & 69.67
 & 61.74 & 85.79 & 84.63
 \\ \hline 
 \end{tabular} \end{centering} \end{adjustbox} \caption{ Taux de réussite détaillés pour le perceptron Perceptron nodistrib} \end{figure} 
Nous remarquons que la plus grosse différence entre les deux consiste
en le score de réussite dans la classification
des mots hors vocabulaire, mais aussi et surtout que le score général 
des mots hors-vocabulaire est très bas par rapport aux autres
classifieurs, dont celui du TP5 \\
Etant donné que toutes les features du TP5 sont présentes dans ce
perceptron, l'explication la plus logique est que les features
ajoutées ne sont pas utiles, et que par conséquent en ajoutant ces
features on complexifie les calculs, ce qui cause cette baisse de la
précision , spécifiqument sur les calculs des mots hors vocabulaire.
\\
Cela est montré par le fait que le perception possédant les
\emph{distributional features} est légèrement moins précis, notamment
sur les mots hors vocabulaire. \\

\begin{figure}[H] \begin{adjustbox}{width=\textwidth} \begin{centering} \begin{tabular}{ | l || *{ 6}{c|c|c||} } \hline 
& \multicolumn{3}{|c|}{ ftb } & \multicolumn{3}{|c|}{ gsd } & \multicolumn{3}{|c|}{ partut } & \multicolumn{3}{|c|}{ pud } & \multicolumn{3}{|c|}{ sequoia } & \multicolumn{3}{|c|}{ spoken }  \\ \hline 
& OOV & AMB & GEN & OOV & AMB & GEN & OOV & AMB & GEN & OOV & AMB & GEN & OOV & AMB & GEN & OOV & AMB & GEN   \\ \hline \hline 
foot  & 33.39 & 78.41 & 67.50
 & 22.68 & 76.55 & 68.47
 & 35.72 & 74.06 & 56.63
 & 35.72 & 74.06 & 56.63
 & 30.69 & 80.58 & 65.61
 & 43.07 & 71.03 & 57.42
 \\ \hline 
ftb  & 74.17 & 91.38 & 94.13
 & 76.99 & 88.72 & 90.89
 & 66.69 & 87.56 & 85.34
 & 66.69 & 87.56 & 85.34
 & 73.49 & 89.68 & 90.02
 & 51.32 & 73.40 & 66.45
 \\ \hline 
gsd  & 67.06 & 86.07 & 88.95
 & 81.07 & 91.52 & 94.23
 & 68.01 & 89.22 & 85.64
 & 68.01 & 89.22 & 85.64
 & 72.79 & 90.68 & 89.69
 & 52.39 & 76.40 & 67.92
 \\ \hline 
natdis  & 23.71 & 84.23 & 77.32
 & 29.57 & 84.79 & 75.75
 & 46.81 & 82.26 & 68.55
 & 46.81 & 82.26 & 68.55
 & 47.40 & 83.04 & 74.63
 & 39.65 & 76.16 & 60.66
 \\ \hline 
partut  & 72.11 & 86.06 & 90.09
 & 73.91 & 90.60 & 90.69
 & 64.84 & 91.75 & 91.96
 & 64.84 & 91.75 & 91.96
 & 74.44 & 90.70 & 89.38
 & 51.45 & 74.38 & 69.34
 \\ \hline 
pud  & 57.61 & 79.01 & 83.57
 & 69.22 & 82.05 & 87.52
 & 63.33 & 78.32 & 79.83
 & 63.33 & 78.32 & 79.83
 & 67.68 & 79.92 & 83.21
 & 50.98 & 76.92 & 66.44
 \\ \hline 
sequoia  & 71.21 & 88.30 & 90.63
 & 75.84 & 90.99 & 92.49
 & 69.14 & 90.62 & 86.17
 & 69.14 & 90.62 & 86.17
 & 77.53 & 93.17 & 94.68
 & 51.87 & 76.88 & 67.82
 \\ \hline 
spoken  & 51.10 & 68.74 & 77.67
 & 48.26 & 75.71 & 82.27
 & 60.32 & 74.82 & 77.24
 & 60.32 & 74.82 & 77.24
 & 59.58 & 72.82 & 78.74
 & 73.97 & 83.31 & 89.31
 \\ \hline 
 \end{tabular} \end{centering} \end{adjustbox} \caption{ Taux de réussite détaillés pour le perceptron SKLearn Perceptron} \end{figure} 

Ici, les résultats sont légèrement meilleurs que le perceptron du
TP5. Cela est plausiblement dû soit à une meilleure implémentation de
l'algorithme (ou une meilleure extraction des features avec
{\verb DictVectorizer } ).
ou à un nombre d'epoch plus élevé. 


L'amélioration la plus sensible par rapport au TP5
est le meilleur score sur les mots
hors-vocabulaire. 
\\

\begin{figure}[H] \begin{adjustbox}{width=\textwidth} \begin{centering} \begin{tabular}{ | l || *{ 6}{c|c|c||} } \hline 
& \multicolumn{3}{|c|}{ ftb } & \multicolumn{3}{|c|}{ gsd } & \multicolumn{3}{|c|}{ partut } & \multicolumn{3}{|c|}{ pud } & \multicolumn{3}{|c|}{ sequoia } & \multicolumn{3}{|c|}{ spoken }  \\ \hline 
& OOV & AMB & GEN & OOV & AMB & GEN & OOV & AMB & GEN & OOV & AMB & GEN & OOV & AMB & GEN & OOV & AMB & GEN   \\ \hline \hline 
foot  & 36.30 & 83.83 & 70.94
 & 23.19 & 79.04 & 70.02
 & 38.49 & 75.80 & 61.56
 & 38.49 & 75.80 & 61.56
 & 33.59 & 82.43 & 67.77
 & 30.08 & 74.43 & 51.53
 \\ \hline 
ftb  & 83.78 & 95.06 & 96.71
 & 80.84 & 91.87 & 92.91
 & 76.79 & 89.63 & 89.38
 & 76.79 & 89.63 & 89.38
 & 79.38 & 92.08 & 92.52
 & 44.66 & 76.24 & 64.92
 \\ \hline 
gsd  & 73.93 & 90.46 & 92.16
 & 85.06 & 94.59 & 96.16
 & 77.35 & 90.92 & 89.55
 & 77.35 & 90.92 & 89.55
 & 78.46 & 92.75 & 92.25
 & 47.49 & 79.27 & 67.43
 \\ \hline 
natdis  & 25.36 & 88.78 & 80.16
 & 43.58 & 87.86 & 79.65
 & 51.51 & 83.22 & 73.80
 & 51.51 & 83.22 & 73.80
 & 50.88 & 85.10 & 77.10
 & 30.18 & 81.49 & 57.16
 \\ \hline 
partut  & 85.57 & 91.15 & 93.71
 & 77.17 & 93.26 & 92.92
 & 75.76 & 94.72 & 95.06
 & 75.76 & 94.72 & 95.06
 & 82.64 & 92.47 & 91.92
 & 51.83 & 76.85 & 71.05
 \\ \hline 
pud  & 63.01 & 82.10 & 86.08
 & 69.06 & 84.31 & 88.90
 & 71.48 & 79.61 & 83.47
 & 71.48 & 79.61 & 83.47
 & 72.09 & 81.61 & 85.46
 & 48.04 & 79.47 & 66.65
 \\ \hline 
sequoia  & 78.73 & 92.61 & 93.90
 & 78.04 & 94.75 & 94.66
 & 75.93 & 92.71 & 89.46
 & 75.93 & 92.71 & 89.46
 & 82.75 & 95.22 & 96.39
 & 48.85 & 78.55 & 67.62
 \\ \hline 
spoken  & 54.74 & 74.08 & 81.35
 & 51.22 & 79.55 & 84.67
 & 63.80 & 78.14 & 80.50
 & 63.80 & 78.14 & 80.50
 & 66.89 & 78.20 & 83.32
 & 81.54 & 89.09 & 92.88
 \\ \hline 
 \end{tabular} \end{centering} \end{adjustbox} \caption{ Taux de réussite détaillés pour le perceptron SKLearn SVM} \end{figure} 

Nous notons des performances légèrement meilleures partout pour le SVM
que pour le Perceptron (l'implémentation de SKLearn), malgré le même
set de features.

Cela est logique, puisque le premier est un
algorithme plus évolué que le second, qui par conséquent permet une
meilleure précision.




%%%  
%D'après nos résultats, nos modèles ont un résultat pratiquement
%similaire avec seulement 3\%  de différence entre le plus mauvais modèles, 
%Perceptron nodistrib, et le meilleur, SKLearn SVM.
%Nous remarquons par ailleurs les modèles perdent du résultats en
%majeur partis à cause du  corpus foot, 
%en effet le corpus foot ne possède que très peu de vocabulaires
%similaires aux autres corpus dû à 
%son vocabulaire spécifique avec en moyenne moins de 40\% de
%vocabulaire reconnus.  Ce qui permet d'avoir des résultats à peine
%bons avec une moyenne  générale des modèles autours des 60\%
%seulement. Quant à natdis , notre second corpus sans trains, il
%possède aussi un vocabulaire  spécifique puisque moins de 50\% du
%vocabulaire dans le test   est reconnus. Cependant, il possède de
%meilleurs résultats généraux  que foot avec une moyenne autour des
%70\%. On peut en déduire  que le vocabulaire spécifique est
%généralement peu retrouvé et   donc que l'OOV et les résultats sont en
%corrélation, en effet  lorsque l'on regarde d'autres résultats comme
%spoken.train  et sequoia.test avec un OOV de 40\% en moyenne,
%le résultat n'est autour que des 60\% pour les modèles les moins bons. 
 
