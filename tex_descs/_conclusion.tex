\section{conclusion}

Pour conclure, nous pouvons dire que nos perceptrons, même si possèdant de bons résultats, sont décevants, en effet même si coder dans le cas spécifique du projet, il reste moins bon que les perceptrons génériques proposés par SKLearn comme dit précédemment surement dû à une meilleur implémentation ou à un nombre d'époch bien plus élévé que les notres. Sinon dans l'ensemble, le projet s'est bien passé hormis quelques problèmes sur la formule de KL-Divergence mais il nous a permis de bien comprendre le fonctionnement et le but d'un perceptron ainsi que ces limites, ici montrées grâce vocabulaire "spécifique" de Twitter. 
